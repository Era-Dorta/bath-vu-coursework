\documentclass[12pt]{article}
 
\usepackage[margin=1in]{geometry} 
\usepackage{amsmath,amsthm,amssymb,bm}
\usepackage{graphicx}
 
\begin{document}
  
\title{Visual Understanding 1 Coursework}
\author{Garoe Dorta-Perez\\
CM50248: Visual Understanding 1\\
Unit Leader: Peter Hall\\
Bath University}
 
\maketitle
 
\section{Introduction}

The main objectives of this unit are to acquire the fundamentals of Computer Vision.
In order to gather this skills, we will attempt to build a stereo reconstruction system.
Using a bottom-up approach the following steps will be taken:
\begin{enumerate}
\item Image convolution, transforming using kernel matrices.
\item Feature construction, detecting scale invariant interest points in images.
\item Matching, using the previous points to match two images.
\item Stereo reconstruction, rebuilding 3D points from 2D matched points in stereo images.
\end{enumerate}

\section{Literature review}


\section{Convolution}

Transforming images has artistic and practical applications, for instance blurring certain parts in a person face to hide defects or detecting borders with differential operators.
For simplicity we will restrict ourselves to gray images, defined as two-dimensional $n \times m$ matrices where each element represents the intensity, and is in the range $\lbrace  0, 1 \rbrace$ where 0 is black and 1 is white.
A convolution kernel $k$, is defined as a $k_1 \times k_2$ matrix, such that $k_1 \leqslant n$ and $k_2 \leqslant m$.
When an image $I_0$ is convoluted with a kernel, a new image $I_1$ is generated, such that each element in the second image is calculated as follows WRITE IT AS EQUATION OR EXPLAIN IT OTHERWISE.
The advantage in this definition is that, it gives a common framework for every possible transformation; so we only need to change the kernel to fit our needs.
However, we need to specify what will happen to the borders of the matrix, leave them as they are, set them to zero, compute circularly, etc; and where does the convolution start with kernel with an even number of columns or rows.

An interesting property of convolutions is the \emph{convolution theorem}, which states that $g * h = F^{-1} ( F(g) \cdot F(h) ) $.
This property allows to transform the images into the Fourier domain, apply the convolution, transform back and obtain the same results with less computational cost.
Below we show some examples of our convolution code as well as a table comparing the performance of our initial code, our code with the Fourier transform and Matlab's implementation.

\end{document}

\section{Results}
